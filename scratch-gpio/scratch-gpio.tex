\documentclass[xetex,table]{beamer}

\usepackage{fontspec}
\usepackage[autostyle]{csquotes}
\usepackage{hyperref}
\usepackage{color}
\usepackage{setspace}
\usepackage{listings}

\usetheme{Pittsburgh}
\usecolortheme{beaver}

\title{Linux e Circuiti elettronici}
\subtitle{con Scratch e Raspberry Pi}
\author{Luca Ceresoli}
\date{}

\AtBeginSection[]{
  \begin{frame}{}
    \huge
    \begin{center}
      \insertsection
    \end{center}
  \end{frame}
}

\begin{document}

\maketitle

\section{Circuiti elettrici}

\begin{frame}
  \frametitle{Circuiti idraulici}
\end{frame}

\begin{frame}
  \frametitle{Circuiti elettrici}
\end{frame}

\begin{frame}
  \frametitle{Un diodo LED}
\end{frame}

\begin{frame}
  \frametitle{Un semaforo}
\end{frame}

\begin{frame}
  \frametitle{Laboratorio}
  \begin{center}
    \LARGE
    Facciamo funzionare il semaforo!
  \end{center}
\end{frame}

\section{Automazione}

\begin{frame}
  \frametitle{Come lo comando automaticamente?}
\end{frame}

\begin{frame}
  \frametitle{Che cosa è un computer?}
\end{frame}

\begin{frame}
  \frametitle{Ma... quale computer?}
\end{frame}

\begin{frame}
  \frametitle{Raspberry Pi 3}
\end{frame}

\section{Un semplice semaforo con Scratch}

\begin{frame}
  \frametitle{Schema di collegamento}
  TODO: schema di collegamento
\end{frame}

\begin{frame}
  \frametitle{Comandare i GPIO con Scratch}
  TODO: gpioserver e broadcast
\end{frame}

\begin{frame}
\frametitle[Lab! Verde, giallo, rosso!]{Laboratorio}
  \begin{center}
    \LARGE
    Verde, giallo, rosso!

    (TODO: loop infinito)
  \end{center}
\end{frame}

\section{Un semaforo intelligente}

\begin{frame}
  \frametitle{Schema di collegamento}
  TODO: schema di collegamento con PIR
\end{frame}

\begin{frame}
  \frametitle{Leggere i GPIO con Scratch}
\end{frame}

\begin{frame}
\frametitle[Lab! Semaforo intelligente!]{Laboratorio}
  \begin{center}
    \LARGE
    Semaforo intelligente
  \end{center}
\end{frame}

\section{E poi?}

\begin{frame}
  \frametitle{E poi?}
  TODO: disegno con frecce: semaforo più complesso, linguaggi ``classici'', arduino...
\end{frame}

\begin{frame}
  \frametitle{Fine}

  \begin{center}
    {\Huge Domande?}

    \vspace{0.1\textheight}

    \href{mailto:luca@lucaceresoli.net}{luca@lucaceresoli.net}\\
    \url{http://lucaceresoli.net}
  \end{center}
\end{frame}

\begin{frame}
  \frametitle{CC BY-SA}
  \begin{center}
    \textcopyright{} Copyright 2016, Luca Ceresoli\\

    \vspace{0.05\textheight}

    \small
    Materiale rilasciato sotto licenza\\
    Creative Commons Attribution - Share Alike 3.0 \\
    \url{https://creativecommons.org/licenses/by-sa/3.0/} \\
  \end{center}
\end{frame}

\end{document}
